%
% Complete documentation on the extended LaTeX markup used for Insight
% documentation is available in ``Documenting Insight'', which is part
% of the standard documentation for Insight.  It may be found online
% at:
%
%     http://www.itk.org/

\documentclass{InsightArticle}


%%%%%%%%%%%%%%%%%%%%%%%%%%%%%%%%%%%%%%%%%%%%%%%%%%%%%%%%%%%%%%%%%%
%
%  hyperref should be the last package to be loaded.
%
%%%%%%%%%%%%%%%%%%%%%%%%%%%%%%%%%%%%%%%%%%%%%%%%%%%%%%%%%%%%%%%%%%
\usepackage[dvips,
bookmarks,
bookmarksopen,
backref,
colorlinks,linkcolor={blue},citecolor={blue},urlcolor={blue},
]{hyperref}
% to be able to use options in graphics
\usepackage{graphicx}
% for pseudo code
\usepackage{listings}
% subfigures
\usepackage{subfigure}


%  This is a template for Papers to the Insight Journal. 
%  It is comparable to a technical report format.

% The title should be descriptive enough for people to be able to find
% the relevant document. 
\title{Sliding windows for fast rank and mean filters}

% Increment the release number whenever significant changes are made.
% The author and/or editor can define 'significant' however they like.
\release{0.00}

% At minimum, give your name and an email address.  You can include a
% snail-mail address if you like.
\author{Richard Beare}
\authoraddress{Richard.Beare@ieee.org}

\begin{document}
\maketitle

\ifhtml
\chapter*{Front Matter\label{front}}
\fi


\begin{abstract}
\noindent
% The abstract should be a paragraph or two long, and describe the
% scope of the document.
The performance of a variety of useful linear and nonlinear filters
can be greatly improved using separable or recursive
implementations. This article presents accelerated implementations of
mean and rank filters using well known recursive formulations. Masked
versions of the rank filters, as well as separable approximations are
also provided.
\end{abstract}

\tableofcontents

\section{Introduction}
The mean and median filters in ITK are based on neighborhood iterators
leading to complexities of $O(n^d)$ for operations on a $d$
dimensional cubic kernel for the mean filter and something similar,
but dependent on the sorting algorithm, for the median filter. 

Recursive implementations of both filters are well known and improve
performance to approximately $O(n^{d-1})$.

The filters provided in this article use the sliding window
architecture from the consolidatedMorhpology article.

\section{Basic theory}
The basic theory behind the algorithms is very simple and won't be
described rigorously here. The basic concept is to improve performance
by eliminating redundancy between adjacent pixels. The algorithm for
the mean filter is:
\begin{itemize}
\item Consider neighboring pixels $i$ and $j$ and kernel $K$. $P_i$ 
      denotes the set of pixels covered by the kernel at location $i$.
\item Compute the sum of pixels in set $P_i$: $S_i = \sum P_i$
\item Compute the sets $I = P_i \ P_j$ and $O = P_j \ P_i$. These are 
      the sets of pixels that are not in common between neighboring kernels.
\item It is now possible to compute $S_j$ as follows: $S_j = S_i + \sum I - \sum O$
\item Output values are simply computed by dividing the sum by the kernel size.
\end{itemize}

The same approach can be used to compute a variance, by maintaining
the sum of squared pixel values.

Rank filters require a more complex approach. Rather than updating a
simple summary statistic, such as the sum, it is necessary to update a
more complex ordered data structure that allows the rank of interest
to be accessed quickly. Histograms are the obvious choice. The rank
filters in this article use a vector based histogram for 8 and 16 bit
data types and a c++ map for larger types. This idea was first
introduced by Huang, 79.

\section{The filters}
The filters provided in this article are:
\begin{itemize}
\item {\bf itkMovingHistogramRankImageFilter}
\item {\bf itkMovingWindowMeanImageFilter}
\item {\bf itkMaskedMovingHistogramRankImageFilter}
\item {\bf itkFastApproxRankImageFilter}
\item {\bf itkFastApproxMaskRankImageFilter}
\end{itemize}

itkMovingHistogramRankImageFilter, itkMovingWindowMeanImageFilter and
itkMaskedMovingHistogramRankImageFilter all support arbitary kernel
shapes and any scalar data type. Usage is fairly standard for this
class of filter, and tests illustrating their usage are available in
the associated archive. itkMaskedMovingHistogramRankImageFilter is a
slightly unusual filter that computes kernel ranks with restrictions
defined by a mask. The output for pixel $i$ is defined as the median
(or other rank) of the set of pixels defined by the intersection of
the mask and $K_i$.

itkFastApproxRankImageFilter and itkFastApproxMaskRankImageFilter are
experiments with separable implementations of rank filters. Rank
operations are not separable, but the difference may be irrelevant for
some applications. These filters apply a series of one dimensional
kernels (one for each dimension of the image) in a cascade to
approximate a rectangular kernel. This approach has a complexity that
is essentially independent of kernel size and therefore may be useful
in applications requiring large kernels.

Separable mean filters can also be implemented using the same approach
and are not approximations.

\section{Application specific details}
\subsection{Boundary conditions}
These filters implement different boundary conditions to the
equivalent ITK filters. The boundary conditions crop the kernel as it
approaches the edge. Therefore the outputs are different to the
existing ITK filters in the border region.

\subsection{Improvements}
The code used for sliding windows is general purpose. Performance
could be improved further by developing code restricted to line
operations. A faster approache to rank filtering employing more
complex decomposition has been described recently, but the
implementation appears very tedious. I'd recommend testing whether
separable approximations are appropriate for your application before
heading down that route.

\appendix



\bibliographystyle{plain}
\bibliography{InsightJournal}
\nocite{ITKSoftwareGuide}

\end{document}

